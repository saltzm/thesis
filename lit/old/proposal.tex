\documentclass{article}
\begin{document}

\title{A Search for Common Substructures within Neural Networks}
\author{Matthew Saltz}
\date{January 2013}
\maketitle

\section{Introduction}
-what are neural networks\\
-how are they used\\
-remark that typically, they are treated as a black box; we want to dive into the box
\section{Proposal}

\subsection{Research Question}

-neural network represented by a graph\\
-neural networks take on different structures:  the node structure set up by the developer and the weight structure formed through training\\
-want to explore whether recognizers for different objects exhibit common weighting substructures\\
-also possible to explore whether recognizers for the same object with different node structures have substructures in common\\
-explore whether there may be a common structure pre-convergence for objects of similar types. E.g., a facial recognition network that recognizes mammal faces before training in specifically on humans or cats, etc.

\subsection{Experimentation}

-train neural networks with various structures on various training sets\\
-use various graph comparison techniques on the resulting networks\\

\section{Significance}

-if there is, for example, a common weighting substructure between visual recognizers, it would be possible to use it to speed up the training process
-

\section{Conclusion}

\end{document}
